\documentclass[12pt, book]{article}
\usepackage{graphicx} % Required for inserting images
\usepackage{hyperref}
\usepackage{float}

\title{CS Portfolio Report}
\author{\textbf{Ahmed Mohammed}}
\date{4, July 2024}

\begin{document}

\maketitle
\centerline{CS Portfolio: \href{https://akemas.github.io/}{Akemas}}
\newpage

\section{Introduction}
\paragraph{This report is to showcase my experiences during the making of my CS Portfolio. Reasons for certain designs, references and improvements will be elaborated.}

\section{Goal}
\paragraph{The goal was to make a website that can showcase my current abilities by making it from scratch and learn along the way. Implementing these skills should demonstrate necessary skills for the job like fast learning and flexibility. }

\section{Design}
\paragraph{There was a lot of thought process going through the design. At first, Black and white professional themed website was in mind to be made. After thorough thinking, I wanted to make it more personalised, a design that can speak many things about me. For example, I wanted to give the impression that I do not fear trying new things and experimenting but with a touch of mine. So, instead of the classic blue black or white designs, I wanted to go for a unique background colour and that's when I chose a tetrad colour between green and blue. It is one of my favourite colours, a mix of calm and energetic vibe.}

\section{Content}
\paragraph{For what valuable information I wanted to add, I did not have enough experience to showcase a lot of work. That's why I focused on the things I do best! making top level PowerPoint slides as well as making websites to learn how to implement new feature into them. I can also talk 3 languages which makes me a versatile cadet for an international environment.}

\section{Code implementations and features}
features and commands I used that was crucial for making this cs portfolio:
\begin{itemize}
    \item Nav to navigate through and a href to make that transition when click about or contact. I also used the basic ul to make an unlisted order.
    \item Div and class was used to define and classify data into different categories to make it easier to edit in the css file.
    \item Onclick, Id(to make an html unique with the actions) and the conditional for in Java and was used to make that cool transition between the skills, experience and education tab.
\end{itemize}

\section{Conclusion and Improvements}
\paragraph{While the website was clean and minimal, It feels like it could have been better with a display of my older works as well as new features like for example, a way to navigate to my social platforms. I plan to add these these features as well as gain new experiences in order to add them in the future.}

\section{References}
Note: There were not much references aside from some youtube videos to help in debugging.
\begin{itemize}
    \item How to change font in VS Code Tutorial (2023). Available at:\\ https://www.youtube.com/watch?v=JzkU9u-DVTk (Accessed: 4 July 2024).
    \item Java in Visual Studio Code (no date). Available at:\\ https://code.visualstudio.com/docs/languages/java (Accessed: 4 July 2024).
    \item textblog(2007), 9 July. Available at: \\https://texblog.org/2007/07/09/documentclassbook-report-article-or-letter/ (Accessed: 4 July 2024).

\end{itemize}



\end{document}
